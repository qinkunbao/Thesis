\chapter{Conclusion and Future Work}\label{chapter6}
This dissertation research studies the address-based side-channel leakages in two aspects: detection and quantification. In the side-channel vulnerability detection research, we explore and solve the two bottlenecks in today's automated side-channel vulnerability detection tool, performance and precision. In the side-channel vulnerability quantification research, we propose two different side-channel leakage quantification tools based on information theory. 

In Chapter~\ref{chapter3}, we present a practical address-based side-channel detection tool, \detect{}, that is capable of detecting secret-dependent control-flows and data accesses at the same time. We model side-channel leakage sites as math constraints and use dynamic symbolic execution to generate constraints for possible leakages. Using statistical testing, \detect{} can identify the true leakages from those constraints. Besides, \detect{} analyze the side-channel leakages directly from X86 execution traces. Existing binary analysis tools usually transfer the x86 instruction to intermediate languages (IR) simplify the implementation. However, such designs have expensive overheads and imprecise analysis results. To tackle the problem, we take the engineering efforts and implement the analysis based on Intel developer's manual from scratch. We evaluate \detect{} on popular cryptographic libraries including OpenSSL, mbedTLS, Libgcrypt, and Monocyper. The evaluation results show that \detect{} is three times to one hundred times faster than the state of art tools while finding all the leakages reported by the previous tools. In addition, \detect{} identity new leakages. The new leakages was later confirmed by other researchers and software developers.  

In Chapter~\ref{chapter4}, we present a novel method to quantify address-based side-channel leakage. We quantify the amount of leakage information for a single trace attack. The amount of the leaked information is based on the search space reduced by side-channel attacks. We rely on \detect{} to detect side-channel leakages. After that, we use approximate model counting to quantify the leakage site. The method is implemented in a prototype tool called \tool{}. We show its effectiveness in quantifying side-channel leakage. With the new definition of information leakage that models actual side-channel attackers, quantifying the number of leaked bits helps understand the severity level of side-channel vulnerabilities. The evaluation confirms that \tool{} is useful in estimating the amount of leaked information in real-world applications.

In Chapter~\ref{chapter5}, we propose a fuzzing method to automatically detect and quantify the address-based side-channel leakages automatically. The proposed method can produce the conservative estimation of the side-channel leakage of the deterministic program. As a result, any severe leakage reported by \ctool{} is the true severe leakage. We also study the total effect of two leakages. Our evaluation result on both several libraries and benchmarks show that \ctool{} is effective and accurate in detecting the side-channel leakage.
