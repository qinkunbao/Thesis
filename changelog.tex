\documentclass{article}
\usepackage[utf8]{inputenc}

\title{Response to the defense feedback}
\author{Qinkun Bao}
\date{May 2021}

\begin{document}

\maketitle
Thank you for reviewing my dissertation. I have gone through the details of each comment and revised my dissertation multiple rounds to eliminate any logic flows and fix all the typos and grammar errors. In summary, the revision covers the following aspects.


\begin{enumerate}
\item \textbf{Fuzzing – Why did you use fuzzing?}

Abacus identifies and quantifies side-channel leakages in a single trace. So it suffers from the problem of code coverage.  That is, Abacus misses side-channel leakages in the unexecuted code path. 

I have a few choices to address the limitation. First, I can use static methods (verification or program analysis). Although I am interested in static analysis, I am not a big fan of current static analysis tools, especially for side-channel research. There are many papers on static side-channel analysis. Nearly all of these tools are used to prove the absence of side-channel leakages. However, most of them have high false-positive rates. 

On the other hand, there are many success stories with dynamic tools (e.g., SAGE, AFL, Valgrind) in recent years. In particular, fuzzing is becoming a hot trend these years, especially after DARPA CGC in 2016. The fuzzing method is suitable for a task such as side-channel detection. If a program executes two different code paths or accesses two different variables with two different input values, the program is vulnerable to side-channel attacks. When we report side-channel leakages sites to developers, we can tell them the two inputs. Developers can quickly reproduce the leakage sites with the inputs.  Interestingly, there are a few (not many) related papers. That's why I used fuzzing at the beginning of the project.


\item \textbf{Discuss the limitations of your research
e.g., can only handle address based …
you might want to include some discussion on the points (3), (4), and (5) below.}

The research presented in the dissertation has a few limitations.

a. The methods in my dissertation cannot find all kinds of side-channel leakages. No tool can find all forms of leakages. Even for address-based side-channel leakages, Abacus and Quincunx cannot find all the leakages due to the code coverage problem. Some verification tools can prove the absence of side-channel leakages for some small code snippets. However, to the best of my knowledge, no tools as yet can prove the absence of side-channel leakages in production software.

b. The quantification result only applies to specific threat models. For example, Abacus quantifies the number of leaked bits during one real execution. So it is possible that one leakage site leaks little information in one execution but leaks much information in another execution.

c. The research in the dissertation can only handle address-based side-channel attacks. Many side-channel attacks infer secret data based on other side-channel signals (e.g, timing, EM signals). However, the root cause of these attacks is still the same. That is, the program accesses different addresses when it processes different input secrets. Under the circumstance, our methods can still detect these leakage sites.

\item \textbf{Can you extend your research (especially the quantification part) to include other side channel signals?}

It is possible to extend the quantification part to include other side-channel signals (e.g., Sound, CPU usages, EM signals). The key is to model the noises properly. In my dissertation, we assume that an attacker can have a noise-free observation. While a noise-free observation is possible for some address-based side-channel attacks, the assumption is too ideal for many other side-channel attacks. The good news is that there are some mature methods in information theory to model noises (e.g., Gaussian noise). The noisy-channel coding theorem provides a computable method to estimate the information that can be reliably transferred over a channel between secret data and attackers' observations. From a research project aspect, I hope to have a stronger motivation before starting the project. 
\item \textbf{A bits vs. Q bits. Lower bounds?
       Improve the presentation
       Make the comparison more meaningful
       Etc.}

It is hard to explain Entropy. There is a rumor about entropy between John von Neumann and Claude Shannon. When Shannon developed the concept of information theory, he did not have a good idea to name the measurement of uncertainty in a system. Neumann told Shannon, ``You should call it entropy, for two reasons. In the first place,your uncertainty function has been used in statistical mechanics under that name, so it already has a name. In the second place, and more important, nobody knows what entropy really is, so in a debate you will always have the advantage.'' During my dissertation proposal, everyone does not really understand how we quantify the amount of leaked information in Abacus. In my final oral exam, everyone is confused with the term ``lower bounds'' except my advisor. 

a. Abacus quantifies the amount of leaked information for a given address access behavior. It is a measure of the amount of leaked information that an attacker can get, given that another event has already occurred.  

b. Quincunx defines the amount of leaked information by the empirical entropy of the observations. For a determinism program, channel capacity is the tight upper bound of the empirical entropy of the observation. In the early draft of the paper, I use the term ``conservative''. Dr. Wu thinks ``conservative'' might be a confusing term and suggests me with the term ``lower bound''. That's the reason that I used the term ``lower bound'' during my PhD defense. I have removed the term ``lower bound'' to avoid any confusion in my dissertation. 

c. I agree with Dr. Liu's comments that the quantification metrics in Chapter 4 and Chapter 5 are different. The reason that I compare them together in Table 5.3 is that they have the same dimensional unit (bit) under the dimensional analysis. I have removed the comparison from the evaluation section in the current version.

To address the above concerns, I have made the following revisions.

a. I revised the description in Chapter 5. I avoided the use of term ``lower bound'' and tune my presentation. 

b. I revised Table 5.3. The evaluation section does not compare Quincunx with Abacus in the current version because they are calculated with different methods and have different threat models.
\item \textbf{iid assumption – does it fit your research on side-channel?}


It is true that Quincunx requires the input variables to be independent and identically distributed. That's why I applied a blackbox fuzzing instead of a coverage-guided fuzzing. As long as each time we get a random input from the whole input space (Populations) that is independent of the previous input, the input variables should satisfy the i.i.d. assumption. For AES-128 encryption, each time we randomly generate a 128 bit key from $0$ to $2^{128}$, the sequence of key values follows the i.i.d. assumption.

The situation is a little different for a machine learning library. In my dissertation, we also evaluated Quincunx on a deep learning model. Each time, we randomly select an item from a dataset (Samples). If the dataset is biased, then we cannot get correct results. The good news is that the most well-known data in machine learning is i.i.d. i.i.d. is the foundation of supervised machine learning methods. The assumption of i.i.d. simplify many methods by assuming the data won't change over the time. In my dissertation, we used MINIST, a well-known handwriting digit dataset. We believe the i.i.d. assumption holds under the circumstance.

I have revised my dissertation to include the discussion on i.i.d.


\item \textbf{Some feedback I gave before your defense.}

As suggested by Dr. Wu, I added a new chapter on Discussion and Limitations (Chapter 6), and discussed potential ways and related work to the mitigated reported side (Chapter 2).

\item \textbf{Improve Grammar.}

Dr. Larus helped me a lot in fixing grammar issues. I also revised the dissertation multiple times. In addition, I used Grammarly to proofread my dissertation.


\end{enumerate}

I have incorporated some discussions above into my dissertation. Please let me know if you have any feedback.

\end{document}
