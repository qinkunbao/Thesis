\documentclass{article}
\usepackage[utf8]{inputenc}

\title{Changelog}
\author{Qinkun Bao}
\date{May 2021}

\begin{document}

\maketitle
Thank you for reviewing my dissertation. I have gone through the details of each comment and revised my dissertation multiple rounds to eliminate any logic flow and fix all the typos and grammar errors. In summary, the revisions cover the following aspects.


\begin{enumerate}
\item \textbf{Fuzzing – Why did you use fuzzing?}

Abacus identifies and quantifies side-channel leakages in a single trace. So it suffers from the problem of code coverage.  That is, Abacus misses side-channel leakages in the unexecuted code path. I have a few choices. I can use static analysis (model checking or static program analysis). Although I am interested in the static analysis, I am not a big fan of current static analysis tools, especially for side-channel analysis. There are several papers on static side-channel analysis. But nearly all of them have high false positive rates. My personal experience with Coverity is similar. It is a commercial static program analysis tool but still generates plenty of false positives. 


On the other hand, there are many successful stories 

\item \textbf{Discuss the limitations of your research
e.g., can only handle address based …
you might want to include some discussion on the points (3), (4), and (5) below.}
\item \textbf{Can you extend your research (especially the quantification part) to include other side channel signals?}

It is possible to extend the quantification part to include other side-channel signals. The key is to model the noises (e.g., Sound, CPU usages, EM signals) properly. In my dissertation, I assume that an attacker can have a noise-free observation. While a noise-free observation is possible for some address-based side-channel attacks, the assumption is too ideal for many other side-channel attacks. The good news is that there are some mature methods in information theory to model the noises (e.g., Gaussian noise). The noisy-channel coding theorem provides a computable method to estimate the information can be reliably transferred between the source (confidential data) and the destination (attackers). From a research project aspect, I hope to have a stronger motivation before starting the project. 
\item \textbf{A bits vs. Q bits. Lower bounds?
       Improve the presentation
       Make the comparison more meaningful
       Etc.}
\item \textbf{iid assumption – does it fit your research on side-channel?}

The research in the dissertation 


\item \textbf{Some feedback I gave before your defense.}

As suggested by Dr. Wu, I added a new chapter on Discussion and Limitations, and discussed potential ways and related work to the mitigated reported side.

\item \textbf{Improve Grammar.}

Dr. Larus helped me fix many grammar issues. I have revised the dissertation multiple times. In addition, I used Grammarly to proofread my dissertation.
\end{enumerate}

I have incorporated some discussions into my dissertation. Please let me know if you have any feedback.

\end{document}
