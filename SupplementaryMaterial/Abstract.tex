% Place abstract below.

\vspace{-0.3in}
Side channels are ubiquitous in modern computer systems as sensitive
information can leak through many mechanisms such as power consumption, 
execution timing, and even electromagnetic radiation. Among them, 
address-based side-channel attacks, such as cache-based attacks, 
memory page attacks, and controlled-channel attacks, are especially
problematic as they do not require physical proximity. Hardware 
countermeasures, which usually require changes to the complex 
underlying hardware, are hard to adopt in practice.  On the contrary, 
software approaches are generally easy to implement. While some existing 
tools can detect side-channel leakages, many of these approaches are 
computationally expensive or imprecise. Besides, many such vulnerabilities 
leak a negligible amount of sensitive information, and thus developers 
are often reluctant to address them. Existing tools do not provide sufficient 
information, such as the amount of information leakage through side channels, 
to evaluate the severity of a vulnerability.

In this dissertation, we present methods to detect and quantify 
address-based side-channel vulnerabilities in real-world applications. 
First, a new method to detect address-based side-channel vulnerabilities 
for the binary code is proposed. We examine the bottleneck in the symbolic 
approaches and improve the analysis precision and performance.
Second, we propose a new program analysis method to precisely quantify
the leaked information in a single-trace attack. We model an
attacker’s observation of each leakage site as a constraint and run
Monte Carlo sampling to estimate the number of leaked bits for each
leakage site. Finally, we extend our approach to quantify side-channel 
leakages from multiple trace attacks.  Unlike the previous side-channel 
detection tools, our approach can identify severe side-channel
leakages without false positives. 

We implement the approaches and apply them to popular cryptography libraries.  
The evaluation results confirm that our side-channel detection method 
is much faster than state-of-art tools while identifying all the known leakages 
reported by previous tools.
The experiments also show that 
our side-channel analysis reports quite precise leakage information
that can help developers better triage the reported vulnerabilities.
This dissertation research develops fundamental and practical techniques for precise side-channel 
analysis in software systems. We have also open sourced our research software prototypes.
As a result, developers can use our tools to develop more secure systems and the academic and industry
communities can further advance side-channel analysis on top of our research.