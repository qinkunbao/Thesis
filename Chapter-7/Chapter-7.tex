\chapter{Conclusion}\label{chapter7}
This dissertation research studies address-based side-channel leakages in two aspects: detection and quantification. In the side-channel vulnerability detection research, we explore and solve the two bottlenecks in today's automated side-channel vulnerability detection tool: performance and precision. In the side-channel vulnerability quantification research, we propose two different side-channel leakage quantification methods based on information theory. 

\section{Summary}

In Chapter~\ref{chapter3}, we present a practical address-based side-channel detection tool, \detect{}, that is capable of detecting secret-dependent control-flows and data accesses at the same time. We model side-channel leakage sites as math constraints and use dynamic symbolic execution to generate constraints for possible leakages. Using statistical testing, \detect{} can identify the true leakages from these constraints. \detect{} analyzes the side-channel leakages directly from X86 execution traces. Existing binary analysis tools usually transfer x86 instructions to intermediate languages (IR) simplify the implementation. However, such designs have expensive overheads and imprecise analysis results. To tackle these problems, we take the engineering efforts and implement the analysis from scratch  based on the Intel developer's manual. We evaluate \detect{} on popular cryptographic libraries including OpenSSL, mbedTLS, Libgcrypt, and Monocyper. The evaluation results show that \detect{} is three times to one hundred times faster than the state-of-art tools while finding all the leakages reported by the previous tools. In addition, \detect{} identifies new leakages. The new leakages were later confirmed by other researchers and software developers.  

In Chapter~\ref{chapter4}, we present a novel method to quantify address-based side-channel leakage. We quantify the amount of leakage information for a single trace attack. The amount of the leaked information is based on the search space reduced by side-channel attacks. We rely on \detect{} to detect side-channel leakages. After that, we use approximate model counting to quantify the leakage sites. The method is implemented in a prototype tool called \tool{}. We show its effectiveness in quantifying side-channel leakage. With a new definition of information leakage that models actual side-channel attackers, quantifying the number of leaked bits helps understand the severity level of side-channel vulnerabilities. The evaluation confirms that \tool{} is useful in estimating the amount of leaked information in real-world applications.

In Chapter~\ref{chapter5}, we propose a fuzzing method to automatically detect and quantify the address-based side-channel leakages automatically. The proposed method can produce an estimation of the side-channel leakage of the deterministic program. As a result, any severe leakage reported by \ctool{} is a true severe leakage. We also study the total effect of two leakages. Using statistical hypothesis testing, we are able to determine if the two leakages are independent. If so, we can calculate the sum of the amount of leakage by adding them. Our evaluation result on both several libraries and benchmarks show that \ctool{} is effective and accurate in detecting the side-channel leakage.

In conclusion, this dissertation research develops practical techniques for precise side-channel analysis in software systems. For each proposed method, we implement and evaluate it on real-world software. The evaluation results show the effectiveness and generality of our proposed method.  We also release the prototypes to facilitate future research in the area.
\section{Future Directions}
There are a few potential directions to extend the dissertation research. 

\textbf{Kernel Space Side-channel Vulnerability Detection.} There are several side-channel vulnerability detection tools. Most of the tools are designed to analyze the side-channel leakages in user-level applications such as cryptographic libraries. It is reasonable since many side-channel attacks are used to break cryptographic applications. Recently, researchers also started to focus on side-channel attacks~\cite{cao2016off} inside the kernel space. However, it is non-trivial to apply existing side-channel detection tools in the Linux kernel. Take the method in the dissertation, for example. We will have the following challenges. First, our proposed tools in the dissertation rely on dynamic binary instrumentation frameworks to collect the runtime information. For kernel programs, a similar task can be achieved by some whole system emulators such as BitBlaze~\cite{song2008bitblaze} and S2E~\cite{chipounov2012s2e}. S2E uses a modified Linux kernel and QEMU to trace the kernel components. S2E uses QEMU 1.0, which does not support the latest Linux kernel. The engineering efforts of extending S2E to work on the latest kernel are significant. Second, the kernel has very complicated control-flow graphs compared to cryptography libraries. Some vulnerabilities need to be triggered by specific input from the user space applications or devices. Our tool cannot handle interruptions and exceptions. But they are common inside the kernel. On the positive side, we expect there are plenty of leakage sites inside the kernel. The dissertation provides methods to identify ``severe'' leakages automatically.


\textbf{Compiler-assisted Side-channel Vulnerability Analysis and Mitigation.}
It is promising to mitigate the side-channel leakages in a compiler backend. For example, the conditional execution can be eliminated by using masks. Eliminating side-channel leakages from compilers is not a brand new idea. Coppens et al.~\cite{Coppens:2009:PMT:1607723.1608124} propose several methods to remove all potential leakages that are related to data flow and control flow.  GNU toolchain adds new options (\textsf{-mlfence-after-load}, \textsf{-mlfence-before-indirect-branch}) to mitigate the Load Value Injection (LVI) attack during the compilation. Coppens et al.'s approach has a slowdown with a factor from 2 to 24. The LVI harden binaries generated from GNU toolchains takes 6-10x longer to run.  For example, GNU toolchains insert \textsf{LFENCE} before any possible vulnerable instructions, including all load instructions, indirect branches, and return instructions. We suggest only changing points that are likely to leak sensitive information. Finally, with the tool presented in the dissertation, we can fix only part of the ``severe'' leakage sites.

\textbf{Side-channel Leakage Quantification from Noisy Observations.} It is possible to extend the quantification part to include other side-channel
signals (e.g., sound, CPU usages, EM signals). The key is to model the noises
properly. In this dissertation, we assume that an attacker can have a noise-free
observation. While a noise-free observation is possible for some address-based
side-channel attacks, the assumption is still too ideal for many real side-channel
attacks. The good news is that there are some mature methods in information theory to model some types of noises (e.g., Gaussian noise). The noisy-channel coding theorem~\cite{shannon1948mathematical} provides a
computable method to estimate the information that can be reliably transferred over a channel between the secret data and attackers' observations. 
