% !TEX root = ../YourName-Dissertation.tex

\chapter{Introduction} \label{chapter1}
\section{Side-channels}
A side-channel attack is any attack that allows adversaries to infer sensitive information based on ``side-channel'' information. In 1996, Kocher published the first side-channel attack in the literature~\cite{kocher1996timing}. He demonstrated that an attacker could break cryptosystems such as RSA or Diffie-Hellman by measuring the execution time of a simple modular exponentiation operation. The simple modular exponentiation calculates the remainder of an exponent divided by an integer.  The simple modular exponentiation operation executes faster when the exponent bit is 0 and slower when the exponent bit is 1. As a result, an attacker can infer the whole secret key by measuring the execution time of the operation. Since then, researchers have proposed numerous side-channel attacks which exploit different kinds of ``side-channel'' information, including execution timing~\cite{184415,disselkoen2017prime+,7163050}, power consumption~\cite{kar20178}, electromagnetic radiation~\cite{agrawal2002side,217605}, and even sound~\cite{chari1999towards}. 

A large portion of side-channel attacks is the joint result of the underlying computer architecture and the software stack.  One example is cache channel attacks. The CPU cache is a smaller but faster memory designed to bridge the gap of the access time between the main memory and CPU registers by storing copies of recently accessed data. Because the size of the cache is significantly smaller than the size of the main memory, the CPU may only find part of the data in the cache. Accessing data from the cache is faster than accessing data from the main memory. Such differences can be reflected in the variance in the program execution time. Hence we have cache channel attacks.


Various countermeasures have been proposed to defend against side-channel attacks. Existing countermeasures can be divided into two categories, hardware-based methods and software-based methods. Hardware-level solutions, such as reducing shared resources, oblivious RAM, and transnational memory~\cite{203878,217537,shih2017t,Zhang:2015:HDL:2775054.2694372}, need to adopt new hardware features and change  today's modern complex computer systems, which are hard to adopt in reality. Therefore, a promising and universal direction is software countermeasures, detecting and eliminating side-channel vulnerabilities in existing software code base. For example, when we examine recent side-channel attacks, many of them exploit two kinds of code patterns: \emph{secret-dependent control-flow transfers} and \emph{secret-dependent data accesses}.
\begin{figure}[h]
    \begin{lstlisting}[xleftmargin=.3\textwidth, xrightmargin=.32\textwidth]
int secret[32];
...
while(i>0){
    r = (r * r) % n;
    if(secret[--i] == 1)
        r = (r * x) % n;   
}
\end{lstlisting}
    \caption{Secret-dependent control-flow transfer}
    \label{fig:secret:cf}
\end{figure}

Secret-dependent control-flow transfers happen when the value of secret can affect which branch the code executes. Figure~\ref{fig:secret:cf} shows an example. Depending on the value of \textsf{secret}, the code may or may not execute line 6. An attacker can retrieve some information by observing the execution control-flow.

\begin{figure}[h]
    \begin{lstlisting}[xleftmargin=.3\textwidth, xrightmargin=.2\textwidth]
static char FSb[256] = {...}
... 
uint32_t a = *RK++ ^ \ 
(FSb[(secret)) ^
(FSb[(secret >> 8)] << 8 ) ^
(FSb[(secret >>16)] << 16 ) ^
(FSb[(secret >>24)] << 24 );
...
\end{lstlisting}
    \caption{Secret-dependent data access}
    \label{fig:secret:da}
\end{figure}

Secret-dependent data accesses happen when the value of secret data affect which memory location the code reads or writes. Figure~\ref{fig:secret:da} shows an example. \textsf{FSb} is an array. Depending on the value of \textsf{secret}, the code may read or write different items in the array.

The key intuition is that the above side-channel attacks happen when a program accesses different memory addresses with different sensitive inputs. As shown in Figure~\ref{fig:secret:cf} and Figure~\ref{fig:secret:da}, if a program shows different patterns in control transfers or data accesses when the program processes different sensitive inputs, the program could have side-channel vulnerabilities. Various techniques can be used to exploit the above leakage site at different granularities. For example, cache channels can observe cache accesses at the level of cache sets~\cite{liu2015last}, cache lines~\cite{184415} or other granularities~\cite{yarom2017cachebleed}. Other types of side-channel attacks such as controlled-channel attacks~\cite{7163052}, can observe a memory access at the level of memory pages. So a natural defense approach is to remove the above code patterns in the software.


However, eliminating side-channel vulnerabilities from real-world software code base is not easy. First, side-channels are a low-level problem in computer systems. As a non-functional property, it can be difficult for programmers to reason about or test.  While recent work~\cite{203878,217537,Wichelmann:2018:MFF:3274694.3274741,Brotzman19Casym,236338,182946} can detect many side-channel vulnerabilities, it either suffers from expensive computation costs or high false positives and false negatives.  Second, most vulnerabilities pose a negligible risk. Although some vulnerabilities result in the key being entirely compromised~\cite{184415, aumuller2002fault}, many others are less severe in reality. Therefore, developers are often reluctant to fixing all of them. We need a proper quantification metric to assess the sensitivity of side-channel vulnerabilities, so a developer can efficiently triage them.  Third, many side-channel vulnerable implementations (e.g., the CRT optimization and lookup tables) have better performance. This is a dilemma between the performance and the security.


In summary, it is hard to apply software approaches in practice due to the following reasons.

\begin{enumerate}
    \item It is hard to identify side-channel vulnerabilities in real-world software. In general, many side-channel vulnerabilities are low-level problems. Control-flows or memory accesses may not be side-channel vulnerabilities if they are independent of the secret inputs. Recent work can identify side-channel leakages. But these tools can only work on small programs or need to apply domain knowledge of the target program to simplify the analysis.
    \item Side-channel vulnerabilities are ubiquitous. While some vulnerabilities can result in the key being entirely compromised, many other vulnerabilities are hard to exploit. Moreover, some vulnerable implementations perform better. Recent work can identify thousands of leaked points, but most of them are not patched by developers because these leakages are either false positives or leaked very little information.
\end{enumerate}

In this dissertation, we try to address the above problem from the following two aspects.
\begin{enumerate}
    \item \textbf{Detection.} We propose a fast and precise side-channel detection method. Unlike previous trace-comparison methods, our method has fewer false positives and false negatives. Compared with methods based on symbolic execution and abstract interpretation, our methods can be applied to real-world cryptography libraries directly. Symbolic execution provides fine-grained information, but it is expensive to compute. Therefore, prior symbolic execution work~\cite{203878,236338,Brotzman19Casym} either analyzes only small programs or applies domain knowledge~\cite{203878} to simplify the analysis. We examine the bottleneck of the trace-oriented symbolic execution and optimize it to work on real-world applications.
    \item \textbf{Quantification.} We propose methods to quantify precisely the amount of information leakage that an attacker can retrieve during side-channel attacks. For the first project, we define the amount of leaked information as the cardinality of possible secrets based on an attacker's observation. For the second project, we we quantify the amount of leaked information based on channel capacity. Before an attack, an adversary has a large but finite input space. Whenever the adversary observes a leakage site, they can eliminate some impossible inputs and reduce the input space's size. In an extreme case, if the input space's size reduces to one, an adversary has determined the input, which means all secret information (e.g., the entire secret key) is leaked. By counting the number of inputs~\cite{10.1007/11499107_24}, we can quantify the information leakage precisely. 
\end{enumerate}


\section{Contribution}
This dissertation makes the following contributions. First, we propose a novel method that can detect side-channel vulnerabilities in real-world applications fast and precisely. Second, we propose methods that can quantify precisely each side-channel leakages. In addition, we study two kinds of threat models: single-trace attacks, and multiple-trace attacks. In summary, we make the following contributions:

To address Problem 1, we make the following contributions on detecting side-channel leakages:

\begin{itemize}
    \item We design a symbolic analysis method that can detect secret-dependent control-flow transfers and secret-dependent data accesses. The key idea is to only symbolically execute machine instructions that are relevant to vulnerabilities while executing other instructions natively. We implement the above method in a tool called \detect{}. The evaluation results shows that \detect{} is 3 to 100 times faster than existing tools while identifying all vulnerabilities when we run it on the same library.  We do not need to apply domain knowledge to analyze part of the original program, which simplify the analysis. In addition, \detect{} finds new leakages which are not identified by previous tools. We confirmed these findings with developers.
\end{itemize}

To address Problem 2, we make the following contributions on quantifying side-channel leakages in a single trace attack:


\begin{itemize}
    \item We propose a method that can quantify fine-grained leaked information from side-channel vulnerabilities that result from actual attack scenarios in a single trace attack. Our approach differs from previous ones in that we precisely calculate the amount of leakage. We model the information quantification problem as a counting problem and use a Monte Carlo sampling method to estimate the information leakage.
    \item We implement the method into a tool called \tool{} and apply it to several pieces of real software. \tool{}\footnote{Abacus is open source under the MIT license.  \url{https://github.com/s3team/Abacus}} successfully identifies previous unknown and known side-channel vulnerabilities and calculates the corresponding information leakage. Our results are useful in practice. The leakage estimates and the corresponding inputs that triggers the vulnerability can help developers triage and fix the vulnerabilities.
\end{itemize}

To address Problem 2, we make the following contributions on quantifying side-channel leakages on a multiple trace attack:
\begin{itemize}
    \item We propose a novel method that can detect and analyze the effect of each side-channel leakage site automatically. Our analysis combines the information from both the source code and the run-time execution, which makes the method more effective in finding the leakages. By applying a chi-squared test, the method can find independent leakages. Attackers can combine multiple leakage sites to retrieve more information as well.
    \item We analyze the amount of information leakage from side-channel vulnerabilities based on the channel capacity. The theory can serve as the foundation of the future work on quantifying the information leakage based on the fuzzing.
    \item We implement above methods in a tool called \ctool{} and evaluate the tool with several benchmarks and real-world applications such as OpenSSL, mbedTLS, Libgcrpyt, TinyDNN, and GTK. The results show that our tool can detect and quantify the side-channel leakages effectively. Moreover, the leakage reports given by the \ctool{} can help developers fix the reported vulnerabilities.
\end{itemize}



\section{Dissertation Organization}
The rest of the dissertation is organized as follows. We present the related work of the dissertation in Chapter~\ref{chapter2}. Chapter~\ref{chapter3} presents a novel method to detect side-channel vulnerabilities in binary executables. Chapter~\ref{chapter4} presents a method that can quantify each side-channel leakage in a single trace attack. Chapter~\ref{chapter5} extends the approach and quantifies the side-channel leakage in a multiple-trace attack.
Chapter~\ref{chapter6} and Chapter~\ref{chapter7} discuss limitations of the dissertation research and propose the future work.