% !TEX root = ../YourName-Dissertation.tex

\chapter{Related Work}\label{chapter2}
This dissertation studies address-based side-channel attacks. We review the related work in this chapter. The related work covers the following aspects: side-channel attacks, detections, mitigation, information quantification, and current binary analysis tools.

\section{Side-channel Attacks}
While there are many kinds of side-channel attacks, we focus on address-based side-channel attacks in this dissertation.
Address-based side-channel attacks happen when a program accesses different memory locations as it processes different secrets. In modern computer systems, an attacker may share hardware resources (e.g., Cache, TLB, and DRAM) with the victim~\cite{ge2018survey,szefer2019survey}. An attacker can probe the shared hardware resource to infer the memory accesses of the victim. While the root cause of the attack is the same, there are many approaches, and an attacker can exploit side-channel leakage sites in different methods to retrieve the information at different granularities.

\begin{table}
    \centering
    \caption{Due to the page limit, we only show some representations of the vast side-channel attacks: the shared resources with attackers, the granularity of the attacker can observe, is there any published attacks on non-cryptography libraries, and if the type of attack has been used to exploit multiple leakage sites.}
    \label{table:side_channel_attack}
    \resizebox{\columnwidth}{!}{%

        \begin{tabular}{lllcc}
            \toprule
            Attacks                                           & Shared Resources & Granularity            & Non-Crpyto & Multiple \\ \midrule
            CacheBleed~\cite{yarom2017cachebleed}             & Cache            & Sub Cache Line (<64 B) & \xmark     & \xmark   \\
            CopyCat~\cite{moghimi2020copycat}                 & Page Tables      & Instruction            & \xmark     & \xmark   \\
            Prime + Probe~\cite{liu2015last}                  & Cache            & Cache Set (64-512 B)   & \cmark     & \xmark   \\
            Prime + Abort~\cite{disselkoen2017prime+}         & Cache            & Cache Set (64-512 B)   & \cmark     & \xmark   \\
            Flush + Reload~\cite{yarom2014flush+}             & Cache            & Cache Line (64 B)      & \cmark     & \cmark   \\
            Flush + Flush~\cite{gruss2016flush+}              & Cache            & Cache Line (64 B)      & \cmark     & \xmark   \\
            Controlled-channel Attack~\cite{xu2015controlled} & Page Tables      & Page (4 KB)            & \cmark     & \xmark   \\
            TLBleed~\cite{gras_translation_2018}              & MMUs             & TLB Set (4 KB)         & \cmark     & \xmark   \\
            Page Cache Attacks~\cite{gruss2019page}           & Page Cache       & Page (4 KB)            & \xmark     & \cmark   \\
            \bottomrule
        \end{tabular}
    }
\end{table}


Table~\ref{table:side_channel_attack} shows some representations of recent side-channel attacks. For a detailed overview of recent side-channel attacks, we refer to the following survey paper~\cite{ge2018survey}. We classified those side-channel attacks into two categories: cache-level channels and page-level channels.

\subsection{Cache-level Channel}
A CPU cache is an essential part of modern computer systems. It is a smaller and faster memory that is designed to reduce the average data access costs to the main memory by storing copies of data from the main memory. When a CPU reads data from a memory location, it first checks the cache. If it finds that the cache has a copy of data from the main memory, a cache hit occurs. Otherwise, the CPU needs to read from the main memory, and a cache miss happens.

Cache channels~\cite{yarom2017cachebleed,191010,184415,Osvik2006,liu2015last,184415} rely on the time difference between a cache miss and cache hit. The core-private caches (e.g., L1 and L2) are traditionally set-associate cache. The cache is divided into different cache sets, and each cache set is made up of several cache lines. Typically, the smallest storage unit in a CPU cache is a cache line. Cache channel attacks can observe cache accesses at the level of cache sets, cache lines or other granularities. We introduce two common attack strategies, namely Prime+Probe and Flush+Reload.

\textbf{Prime+Probe.} Prime+Probe targets a single cache set.
An attacker primes the cache set with its own data and waits until the victim executes the program. Then the attacker probes the cache by reading previously loaded data. If the victim accesses the cache set and evicts some lines of the data, the attacker will experience a slow measurement. Under the circumstance, the attacker can know that the victim must have touched some data that map to the same cache set.

\textbf{Flush+Reload.} Flush+Reload allows an attacker to observe memory accesses at the granularity of a cache line instead of a cache set. While Flush+Reload has a fine-grained granularity compared with Prime+Probe, it requires the attacker and victim to share some memory. It also has two stages. During the ``flush'' stage, the attacker flushes the ``monitored
memory'' from the cache and waits for the victim to access the memory, which may be a load of the sensitive information to the cache line.  In the next phase,
the attacker reloads the ``monitored memory''. The attacker's process read all data in the `monitored
memory''. If the time is relatively fast, it indicate that the data has been accessed by the victim program and the cache has loaded the data. 
On X86-64 machines, the two steps can be combined by measuring the time differences of the \textsf{clflush} instruction.

Based on the above attacks, some work~\cite{brickell2011technologies} suggests having secret-dependent memory accesses at finer than the size of one cache line should be secure enough. However, recent studies show that having different memory accesses within a cache line is also dangerous. One example is the CacheBleed attack~\cite{yarom2017cachebleed}. It is a cache channel attack that can exploit cache-bank collisions. 
\subsection{Page-level Channel}
Page channel attacks allow an attacker can observe the memory access at the granularity of memory pages. One example is the controlled-channel attack~\cite{xu2015controlled}. It is a primary threat to the trusted execution environments (TEEs).  A privileged adversary (e.g., kernel) can infer sensitive information by manipulating the enclave page accesses. A malicious OS can proactively revoke a virtual enclave memory page from the virtual memory. If the victim enclave program accesses the memory in the  page, then the OS can observe the page faults and know which page is accessed. While the granularity of the controlled-channel attacks is more coarse-grained compared with cache channel attacks, controlled-channel attacks allow the attacker to have a noise-free observation. Other examples such as TLB side-channel attacks and DDRAM attacks share the same page-level granularity.


Based on the above discussion, we consider three types of granularities in the dissertation: byte (1 Byte), cache line size (64 Bytes), page size (4 KBs). To the best of our knowledge, these three types of granularities can cover most of the side-channel attacks in the literature.


\subsection{Attack Models}
Previous work~\cite{182946} classifies side-channel attacks into three types based on the type of information that an attacker can learn during the attack.
\subsubsection*{Timing-based Attacks}
In timing-based attacks, an adversary can measure and compare the execution time of victim programs. It could be the overall execution time or the execution time of some interesting points (e.g., a function).  The execution time is correlated to some events (e.g., cache hits and misses, page faults, the number of instructions). Timing-based attacks are perilous because these attacks are able to be carried out remotely over the networks remotely~\cite{brumley2005remote}. Examples of these attacks include remote timing attacks~\cite{brumley2005remote}, Bernstein's attack on AES~\cite{bernstein2005cache}, and Kocher’s original attack~\cite{kocher1996timing} from CRYPTO 96.
\subsubsection*{Access-based Attacks}
In access-based attacks, an adversary can learn the memory addresses that are accessed during the execution after the termination of the victim program. 
For example, a cache-based channel attack can know which memory access incurs a cache hit and which memory access incurs a cache miss to infer addresses that the victim program accesses. Since the attacker can probe the victim program after the execution, the attacker does not know the order of memory accesses. Access-based attacks often happen in the cloud computing environment when the attacker and victim share the same hardware (e.g., LLC cache). Examples of those attacks include cache template attacks~\cite{191010}, Prime+Probe attacks~\cite{liu2015last}, and Flush+Reload attacks~\cite{yarom2014flush+}.
\subsubsection*{Trace-based Attacks}
In trace-based attacks, an adversary can interrupt and  probe the victim program at any point. The attacker can learn the order of memory accesses. Suppose the victim program has two functions \textsf{a} and \textsf{b}. In an access-based attacks, the attacker can only learn both \textsf{a} and \textsf{b} are executed. In a trace-based attack, the attacker can know the execution sequence of two functions (e.g., $\mathtt{aabbb}$). Controlled-channel attacks~\cite{moghimi2020copycat, xu2015controlled} belong to this category where a malicious operating system can interrupt the victim application and probe the program at any time. In general, the trace-based attack has the strongest assumption about the threat model,  but it can retrieve more information compared to the other two types of attacks.
\subsection{Target Applications}
There is a vast number of studies\cite{yarom2017cachebleed,191010,184415,Osvik2006,liu2015last} on side-channel attacks. Most early works targeted retrieving encryption keys from cryptography libraries. In recent years, researchers start to focus on more general and diverse applications. In summary,  the target applications of current side-channel attacks can be classified into three categories.

\subsubsection*{Cryptography Libraries}
\begin{table}
    \centering
    \caption{This table shows some common vulnerable points in cryptography implementations and corresponding software mitigation methods.}
    \label{chapter2:table:crypto}
    \resizebox{\columnwidth}{!}{%

        \begin{tabular}{lll}
            \toprule
            Implementation & Common Vulnerable Point             & Mitigation \\ \midrule
            AES &  T-table and S-table lookups   & AES instructions (AES-NI) 
            \\&&Scalar bit-slicing \\&& Preloading tables     \\
            DES & Table lookups & Scalar bit-slicing (No implmentation) \\
            RSA & CRT optimization & Verifying the consistency with the public key\\
            & Modular exponentiation& Montgomery modular multiplication\\
            & & Base blinding and exponent blinding\\
            (EC)DSA & Scalar multiplication& Constant time scalar implementation\\
            &Modular inversion & Blinding\\
            &Modular reduction &\\
            \bottomrule
        \end{tabular}
        }
\end{table}

The first side-channel attack on cryptography libraries can be traced back to Kocher's timing attacks~\cite{kocher1996timing} in 1996. It exploits the simple modular exponentiation algorithm function in many algorithms (Diffie-Hellman, RSA) implementations. The function does not run in a fixed time. By measuring the time differences between the modular exponentiation algorithm computes the exponent bits (0 or 1), Kocher's attack can recover the whole secret key bit by bit. Since then, researchers start to break all kinds of cipher implementation through side-channel attacks.  Table~\ref{chapter2:table:crypto} summarizes some common vulnerable points in cryptography library implementations.  Note that cryptography libraries usually have several implementations for some ciphers.
For example, AES can have more than five different implementations in OpenSSL. Depending on the build configuration and the return value of \textsf{CPUID}, the code may execute any implementation of them. The reference implementation usually adopts T-Tables and S-Tables to speed up the computation. However, such implementations are vulnerable to side-channels and can be exploited to retrieve the whole key~\cite{bonneau2006cache}. In addition to the reference implementation, those libraries also have side-channel mitigated versions. However, recent studies also show those versions (e.g., AES-NI) also have side-channel vulnerabilities caused by compiler optimizations and human mistakes~\cite{217537}. The situation is even worse for asymmetric ciphers (e.g., Diffie-Hellman, RSA, ECDSA). These ciphers have a lot of big number calculations. It is hard to implement those calculations to be completely side-channel resistant. Recent work~\cite{arnaud2013timing,yarom2017cachebleed,yarom2014flush+} has demonstrated several end-to-end attacks to recover the private key of those ciphers. 

\subsubsection*{Machine Learning Applications}
Recent side-channel attacks on machine learning applications mainly focus on recovering two types of sensitive information: the input of the model and the architecture of deep learning models. Hua et al.~\cite{hua2018reverse} were the first in the literature to reverse engineer the architecture of convolutional neural networks (CNNs) through the memory access patterns. Wei et al.~\cite{wei2018know} perform the side-channel attacks on an FPG-based convolution neural network by using the power consumption to recover inputs to the networks. For an image classification task trained by the MNIST dataset, their attacks can achieve up 89\% accuracy. CSI NN~\cite{batina2019csi} reverse engineered the multi-layer perception and convolution neural network based on the electromagnetic signals (EMs). Cache Telepathy~\cite{yan2020cache} introduces the cache-based side-channel attack to extract DNNs' architectures. We find that all existing approaches need a lot of domain knowledge and expertise to launch the attacks.
\subsubsection*{Media and Graphic Libraries}
Recent work also studies the side-channel attacks in media and graphic libraries. Gruss et at.~\cite{191010} show an attack that can infer keystrokes by exploiting secret-dependent memory accesses in the GDK library. In the controlled-channel attack~\cite{xu2015controlled}, Yuanzhong et al. demonstrate an attack to recover the outline of an image by exploiting the vulnerability in the inverse DCT function in libjpeg. Daimeng et al.~\cite{wang2019unveiling} present a cache attack on graphic libraries. Their approaches uses a machine learning approach to pick cache lines that can be exploitable to leak the most information.
\subsubsection*{Others}
Some other side-channel targets also include spelling checking tools and font libraries~\cite{xu2015controlled}. The attack on font libraries is essentially the same as the attack on the graphic library GDK. By exploiting the differences of memory accesses when the program renders different characters, an attacker can recover parts of the original sensitive information. Other side-channel attacks target the OS's kernel. One example is using the side-channel attack to break the kernel address space randomization (KASLR).
Many exploits rely on the knowledge of the memory location of a certain kernel function. Since Linux 4.12, the kernel loads core kernel image and device drivers at random positions during the boot time. Jang et al.~\cite{jang2016breaking} rely on side-channels to reveal the mapping status of each page. After that, they detect kernel modules with unique size signatures to break KASLR. The second example relies on the side-channel vulnerabilities~\cite{cao2019principled} inside kernels to infer the TCP sequence numbers, which can be used by a third party to hijack connections between the client and the server~\cite{cao2016off}.

\section{Defenses and Mitigation}

Both hardware~\cite{Page2005PartitionedCA,
    Wang:2007:NCD:1250662.1250723,Zhang:2015:HDL:2775054.2694372,Li:2014:SLH:2541940.2541947,
    236344, 236334} and software~\cite{shih2017t,Coppens:2009:PMT:1607723.1608124,
    brickell2006software,crane2015thwarting, 197207} side-channels mitigation techniques have
been proposed recently. Hardware countermeasures, including partitioning hardware resources~\cite{Page2005PartitionedCA}, randomizing cache
accesses~\cite{Wang:2007:NCD:1250662.1250723, 236344}, and designing new
architecture~\cite{tiwari2011crafting}, require changes to complex processors and are difficult to adopt. On the contrary, software approaches are
usually easy to implement. Coppens et
al.~\cite{Coppens:2009:PMT:1607723.1608124} uses a compiler
to eliminate key-dependent control-flow transfers. Crane et
al.~\cite{crane2015thwarting} mitigates side-channels by randomizing software.
As for crypto libraries, the basic idea is to eliminate key-dependent
control-flow transfers and data accesses. Common approaches include
bit-slicing~\cite{konighofer2008fast,rebeiro2006bitslice} and unifying
control-flows~\cite{Coppens:2009:PMT:1607723.1608124}.

\subsection{Software Countermeasures}
\subsubsection*{Software Vulnerability Identification}
For many software countermeasures, the first step is identifying side-channel leakages in software manually. It can be done manually or by some program analysis tools.
There are several existing works on side-channel leakage detections. According to Table~\ref{table:side_channel_detection_tool}, existing methods can be classified into two categories, tools based on the trace comparison and tools based on the abstraction.

\textbf{Trace comparison.} If a program has two different execution traces under two different sensitive inputs, then attackers can distinguish the sensitive input by observing the execution trace. Therefore, we can conclude the program is vulnerable to side-channel attacks. The method is similar to the UNIX tool \textsf{diff}.  The approach does not need to model or abstract the program's semantics. In general, the approach is quite fast. On the negative side, comparing the trace can find the leakage, but it cannot find the complex relationship between the leakage site and the original secret input. For example, if two leakage sites leak the same information, these approaches cannot find out the two leakage sites are the same.

\textbf{Abstraction.} The second category of those tools belongs to the field of abstractions. Tools based on these approaches use abstraction to generate approximations to model the semantics of computer programs. After that, by analyzing the model, these tools can identify side-channel leakages. Compared with the trace comparison approach, the method is more precise.

Those methods usually follow the steps below. Suppose a program $P$ has $k$ as the input. Those approaches first retrieve the relationship between the $k$ and temporary values during the execution. After that, they should have a cache or memory access model and model the side-channel leakage as a math constraint. By using SMT solvers to find different secret inputs that lead accesses to different locations in the memory or the cache, those approaches can determine if the program is vulnerable to side-channel attacks.
Unlike to the trace comparison, the method is quite precise. Moreover, it can model multiple leakages precisely as the conjunction of those formulas. However, these approaches have the following limitations.

\begin{itemize}
    \item Many tools can only analyze the source code or the intermediate representation level. However, side-channel are low-level problems in general.  Most existing tools are research prototypes. For some projects, the tools can only handle the benchmarks in their papers properly. 
    \item Those tools heavily rely on SMT solvers. Despite SMT solvers having made incredible progress in performance, SMT solving is still the main bottleneck of those tools. SMT solvers are not good at non-linear arithmetic. However, such non-linear arithmetics are quite common in cryptography algorithms.  Recent studies show existing solvers still have bugs when dealing with real number calculations. Those calculations are also common in media libraries and machine learning applications.
\end{itemize}

According to Table~\ref{table:side_channel_detection_tool}, some methods may choose to analyze the source code or the intermediate representations of the vulnerable software. However, it still cannot solve the problem. First, many widely-used libraries (e.g., Intel MKL) do not open the source code. Second, some libraries use hard-coded assembly code to pursue maximum performance.
Here, we introduce several tools. CacheAudit~\cite{182946} uses abstract domains to compute an
over-approximation of cache-based side-channel information leakage upper bound.
However, it is difficult to judge the sensitive level of the side-channel leakage based on the leakage provided by CacheAudit. Besides, CacheAudit is not scalable enough to analyze ciphers like RSA implementations. CacheS~\cite{236338} improves the approach from
CacheAudit with a new abstract model that only tracks
secret-related code. Like CacheAudit, CacheS cannot
indicate the sensitive level of side-channel vulnerabilities.
CaSym~\cite{Brotzman19Casym} introduces a static cache-aware symbolic reasoning technique to cover multiple paths in a target program. CaSym can work on two threat models: trace-based attacks and access-based attacks. Again, their
approaches cannot evaluate the sensitive level for each side-channel vulnerability, and it only works on small code snippets.

The dynamic approach, usually consisting of taint analysis and symbolic execution,
can perform a very precise analysis. CacheD~\cite{203878} takes a concrete execution trace and runs symbolic execution on the trace
to get a formula for each memory address. CacheD is
quite precise in avoiding false positives. However, CacheD is not able to detect secret-dependent control-flows. We adopted a similar approach to model the secret-dependent data accesses in the dissertation. \tool{} differs from CacheD in that we do not use traditional taint tracking or domain knowledge to cut the trace when identifying secret-dependent data access vulnerabilities. \tool{} works on machine instructions directly instead of intermediate representations. Moreover, \tool{} finds secret-dependent control-flows at the same time and gives a precise quantification of the leakage. DATA~\cite{217537} detects address-based side-channel vulnerabilities by comparing and aligning different execution traces under various test inputs. After collecting execution traces, DATA aligns them and finds the differences. It uses statistical hypothesis testing to find true leakages. However, both imperfect trace alignment and a statistical testing result that DATA can produce false positives.
MicroWalk~\cite{Wichelmann:2018:MFF:3274694.3274741} adopts a similar approach from DATA by comparing all accessed addresses. It uses mutual information (MI) between sensitive input and execution state to quantify the information. 

\begin{sidewaystable}

    %\begin{table*}
    %\centering
    \caption{Many leakage detection tools are only evaluated on cryptography libraries.  All of them can not reason about the total effect of multiple leakages.}
    \label{table:side_channel_detection_tool}
    \resizebox{1\columnwidth}{!}{%

        \begin{tabular}{lllcclc}
            \toprule
            Tool                                                 & S/D     & Method                              & AVX/SIMD & Multiple & Evaluation                                           & Available \\ \midrule
            CacheAudit~\cite{182946}                             & Static  & Abstraction                         & \xmark   & \xmark   & Crypto Algorithm                                     & \cmark    \\
            CacheD~\cite{203878}                                 & Dynamic & Symbolic Execution + Taint Analysis & \xmark   & \xmark   & Libgcrypt, OpenSSL                                   & \xmark    \\
            CachSym~\cite{Brotzman19Casym}                       & Static  & Symbolic Execution                  & \xmark   & \xmark   & Crypto Snippet                                       & \xmark    \\
            CacheS~\cite{236338}                                 & Static  & Abstract Interpretation             & \xmark   & \xmark   & Libgcrypt, OpenSSL, mbedTLS                          & \xmark    \\
            DATA~\cite{217537}                                   & Dynamic & Trace Alignment + Comparison        & \cmark   & \xmark   & OpenSSL, PyCrypto                                    & \cmark    \\
            ctgrind~\cite{langley2010ctgrind}                    & Dynamic & Taint Analysis                      & \cmark   & \xmark   & Crypto Library                                       & \cmark    \\
            Satcco~\cite{xiao2017stacco}                         & Dynamic & Trace Comparison                    & \cmark   & \xmark   & OpenSSL, GnuTLS, mbedTLS, WolfTLS, LibreSSL          & \xmark    \\
            ANABLEPS~\cite{wang2019time}                         & Dynamic & Trace Comparison                    & \cmark   & \xmark   & gsl, Hunspell, PNG, Freetype, QRcodegen, Genometools & \xmark    \\
            MicroWalk~\cite{Wichelmann:2018:MFF:3274694.3274741} & Dynamic & Trace Comparison                    & \cmark   & \xmark   & Intel IPP, Microsoft CNG                             & \cmark    \\
            \bottomrule
        \end{tabular}
    }
    %\end{table*}
\end{sidewaystable}

\subsubsection*{Software Patch}
Once we have identified side-channel leakages, there are two common approaches to mitigate side-channel attacks. The first approach is to adopt unified behavior code when computing the secret data. For example, a control flow leakage happens when different secret inputs lead to different paths. Software developers can mitigate the leakage by executing both branches and select the result with bit-wise operations.   However, for exiting code base, it is hard to migrate all the code into a side-channel free implementation. The second approach is to introduce randomized behaviors during the execution to obfuscate the information that an attacker can retrieve. One example is cryptographic blinding~\cite{coron1999resistance}. Developers can use a random value to blend the secret key but still generate the same encrypted data with a carefully designed algorithm.  Binding values can alter the program execution to some unpredictable states and introduce the randomization to the execution. However, the binding value can also be retrieved by an attacker through other methods, including side-channel attacks. An attacker can launch an attack to retrieve the binding value to recover the secret input.
\subsection{Hardware Countermeasures}
\subsubsection*{Resource Partitioning}
Side-channel attacks usually require attackers and victims to share some hardware resources (e.g., CPU cache, MMU). A straightforward way to defend those attacks is to partition the resources and ensure the attacker and the victim never share the same hardware resource~\cite{Page2005PartitionedCA,shi2011limiting,liu2016catalyst}. For example, in a cloud computing environment, several virtual machines (VMs) can be hosted on the same physical machine and those VMs share the same last-level cache (LLC). Researchers proposed different methods to partition the LLC such as cache coloring~\cite{shi2011limiting}, using Intel Cache Allocation Technology~\cite{liu2016catalyst}.
\subsubsection*{Randomization}
This type of defense randomizes the behaviors of hardware resources to make it harder for attackers to infer secrets based on side-channel information. For example, Random Permutation Cache (PRcache)~\cite{wang2007new} uses a dynamically randomized memory to cache mapping table for each different process. Vattikonda et al.~\cite{vattikonda2011eliminating} add noises to the result of the system timer to mitigate timing-based channels.

\subsection{Discussion on Existing Countermeasures}
Despite a lot of hardware countermeasures that have been proposed in academia, few of them have been adopted in real-world production systems. In general, hardware countermeasures usually require new hardware designs and are harder to adopt in practice.  In addition, some hardware methods (e.g., Oblivious RAM) may introduce expensive computation overhead. Therefore, many practical solutions are still software countermeasures. However, software methods are not very general, and it is hard to find many side-channel leakages in large code bases manually. In this dissertation, we propose side-channel analysis methods to address the problem.
\section{Information Quantification}
Proposed by Denning~\cite{robling1982cryptography} and Gray~\cite{gray1992toward},
Quantitative Information Flow (QIF) aims to provide an estimation of the amount of leaked information from the sensitive information given the public output. If zero bits
of the information are leaked, the program is called non-interference. 
McCamant and Ernst~\cite{McCamantE2008} quantify the information leakage as the network
flow capacity. They model the possible information leakage as a network of limited-capacity
channel. After that, they use the maximum rate of the channel to represent the amount of secret information that the execution can reveal.
Backes et al.~\cite{5207642} propose an automated method for QIF
by computing an equivalence relation on the set of input keys. However,
the evaluation shows the approach only works on small programs.
Besides, their approach cannot handle programs with bitwise operations.
However, such bitwise operations are quite common in cryptographic libraries.
Phan et al.~\cite{Phan:2012:SQI:2382756.2382791} propose symbolic QIF. The goal of their
work is to ensure a program is non-interference. They adopt an over-approximation method to estimate the total information leakage, and their method
does not work for secret-dependent memory access side-channels.
Pasareanu et al.~\cite{pasareanu2016multi} combine symbolic analysis and Max-SMT solving to synthesize the concrete public input that can lead to the worst-case leakage. They assume the target program has multiple different input secrets and calculate the average leakage for one-fixed public input.
CHALICE~\cite{Chattopadhyay:2017:QIL:3127041.3127044} quantifies the leaked
information for a given cache behavior.
It symbolically reasons about cache
behavior and estimates the amount of leaked information based on cache miss/hit.
Their approach only scales to small programs, which limits its usage in
real-world applications. On the contrary, \tool{} assesses the sensitive level
of side channels with different granularities. It can also analyze side channels in real-world cryptographic libraries.

Model counting refers to the problem of computing the number of satisfying
solutions for a propositional formula (\#SAT). There are two approaches to
solving the problem, exact model counting, and approximate model
counting. Exact model counting for propositional formulas (\#SAT) should determine
the number of solutions. Intuitively, such a model counter should traverse the whole
search space to find all the satisfying solutions. Consequently, the exact model counter is very slow. Those model counters have difficulty scaling up to real-world programs.
On the other hand, the approximate model counter is much faster compared with the exact model counter. The related work focuses on approximate model counting since it is close to our approach. Wei and Selman~\cite{wei2005new} introduce
ApproxCount, a local search-based method using Markov Chain Monte
Carlo (MCMC). ApproxCount has better scalability than
exact model counters. Other approximate model counter includes
SampleCount~\cite{gomes2007sampling},
Mbound~\cite{gomes2006model}, and MiniCount~\cite{kroc2008leveraging}.
Unlike ApproxCount, these model counters can give lower or upper bounds with guarantees.
Despite the rapid development of model counters for SAT and some
research~\cite{chistikov2017approximate,phan2015model}, existing tools cannot be directly applied to side-channel leakage quantification.
ApproxFlow~\cite{biondi2018scalable} uses ApproxMC~\cite{chakraborty2016algorithmic} for information flow quantification, but it has only been tested with small programs.

\section{Trace-based Binary Analysis}
Trace-based binary analysis can be seen as a type of dynamic analysis. It analyzes one concrete program path one time rather than trying to cover
as many paths as possible. Compared to static analysis, trace-based analysis
is usually more precise as it can incorporate runtime information. It usually has
two steps. The first step is collecting trace information. Existing
approaches usually often rely on binary instrumentation frameworks or emulators (e.g., Intel Pin~\cite{luk2005pin},
Valgrind~\cite{nethercote2007valgrind}, and QEMU). The second step is analyzing the trace using techniques, such
as symbolic execution, taint analysis. Depending on when the analysis is launched,
there are two kinds of trace-based analysis, online analysis, and offline analysis. As the name suggests, online analysis is performed during the program execution. The target program's data can be modified directly in memory, which is useful to tasks like fuzzing because the analysis
can force the program to execute a specific branch. 
Offline analysis or post-analysis happens after the program finishes executions. 
Trace-based binary analysis has been adopted in several tasks like algorithm
detection, vulnerability detection, and malware analysis. We only mention a few of
the most related works here. Xu et al.~\cite{xu2017cryptographic} introduce bitwise symbolic execution, which can
identify crypto algorithms in obfuscated binaries. Ming et al.~\cite{ming2017binsim} use system call sliced segment equivalence checking to do binary diffing. Jia et al.~\cite{jia2017towards}
perform an offline analysis on execution traces to identify heap overflows.

There are several known binary analysis frameworks that can help users to launch analysis on binary executables. BitBlaze~\cite{brumley2011bap} is a binary analysis platform that provides both static and dynamic analysis. It uses the tracecap plugin to collect execution traces and convert the trace into Vine IR. Binary Analysis Platform (BAP)
is another binary analysis platform. Similar to BitBlaze, it uses a Pintrace tool to
collect execution traces and convert the trace into BAP IR. Angr~\cite{shoshitaishvili2016state}, the most popular
binary analysis platform recently, also supports trace-based analysis. It converts each instruction into VEX IR and interprets each IR in Python. 

The above existing well-known binary analysis frameworks adopt the IR layer, which makes it easy to implement the analysis and support multiple architectures. However, it can have very low efficiency and long traces. All IR must be interpreted no matter if symbolic variables are involved or not. As side-channels are low-level problems in general, the IR will lose some information. For example, IR-level branches are not necessarily a super-set or a sub-set of machine-code level branches. Some IR-level branches can be compiled into non-branch instructions like conditional moves. Those above limitations motivate us to implement a new trace-based binary analysis framework.